\begin{conclude}
 \chapter{Conclusions~}
 \hyperlink{toc}{Return to TOC}
 
  This thesis has highlighted the significance of ion solvation across the biological, chemical, and physical spectrum of science. I began the discussion
  with a review of Hofmeister's observations noting the varying precipitating capacity of different salts on egg white proteins. From this, I developed
  the narrative that explaining these effects from a modeling perspective was simply too difficult and the conclusions of different experiments often
  conflicted with one another. This in turn motivated study of the simplest case, where building a model for more complex systems might be more approachable.
  The target of these simpler studies is the determination of a single-ion solvation free energy scale from which other properties such as activities, 
  osmotic coefficients, surface tensions increments, and surface potential increments might also be established. The challenge in addressing this scale 
  was found to be two-fold: 1) it is unclear what level of theory is appropriate for handling ion solvation problems and 2) experimental determination of
  the bulk values of these properties is impossible without the use of a \emph{hopefully} reasonable assumption. As regards the first challenge, I reviewed
  an extensive body of literature which found non-electrostatic contributions evaluated at the electronic structure level and the granularity of all-atom 
  models to be necessary to achieve spectroscopic/thermochemical accuracy. For the second, I made the case that the generally binary split of experimental
  single-ion scales was linked by the surface potential of water and that such potentials would be present for other systems as well.
  
  In Chapters \ref{ch3:sec0:level1} and \ref{ch4:sec0:level1} I discussed my own contributions addressing the first of these challenges. I used quantum
  chemistry calculations and an energy partitioning scheme on small alkali/water and halide/water clusters to divvy up the interaction energy into chemically
  informative contributions. These fragments included non-electrostatic pieces like polarization, charge transfer, and dispersion. I discovered that in anion/water
  clusters, these contributions accounted for about 33\% of the total attractive part of the interaction energy. The rest was due to electrostatics. While
  dispersion was found to be appreciably less important in cation/water clusters, polarization played a significant role as well. The electronic charge 
  assigned to individual atoms can be partitioned as well. The charge on the ion was monitored with increasing hydration number. With \emph{n} = 6 waters
  attached to the ion, some 20\% of the excess charge of anions was shared with neighboring waters. A visual of this was provided as well, outlining poles
  in the electron density of the anion directed at each of the waters. It was found, in agreement with previous studies, that the non-electrostatic 
  interactions saturated or nearly saturated within the first solvation shell. This implies that simpler electrostatic models of ion solvation are suitable
  for handling longer-ranged aspects of ion solvation, greatly reducing the computational cost of calculating accurate properties.
  
  The focus on local interactions was extended in Chapter \ref{ch4:sec0:level1} to the non-aqueous solvents ethylene and propylene carbonate which are
  used in Li-ion batteries and supercapacitors. Ion solvation is especially important in the continued evolution of energy storage materials with solvation
  free energies an indicator of an electrolyte's ion transport properties\cite{husch2015large}. Very little free energy work has been done with these solvents
  despite their presence in our everyday lives through cell phones, laptops, and many other devices. Previous simulation results from our group suggested classical
  force fields lacked the flexibility to properly mimic polarization, with a standard model underestimating the effect and a corrected model overestimating it.
  \emph{Ab initio} molecular dynamics simulations were performed for the ethylene and propylene carbonate solvents. I compared the average dipole moment
  of these solvents in the gas phase, liquid phase, and solvating a single Li\sur{+} ion. The induced dipole moment of a molecule in a field is related to 
  the polarizability of the molecule. Relative to the gas phase, I found the dipole moment of each of these solvents increased 34\% over the gas phase
  average when simulating the liquid phase and slightly over 50\% for molecules in the first solvation shell of the ion. This is odd behavior compared to
  water where it's been found that the water dipole moment near ions is nearly the same as in the bulk. This study is not yet complete, but I have already
  learned that an explicit treatment of polarizability in these solvents is essential for modeling the local solvation structure near ions. I hope to also
  determine the strength of the induced dipole-induced dipole repulsion interaction between the first shell molecules and relate it back to our initial
  study of the thermodynamic data.
  
  The last pair of chapters of my thesis dealt with the surface potential of water and the implications of my interpretation of this potential on force
  field development and molecular simulation in general. In Chapter \ref{ch5:sec0:level1}, I used a novel approach involving the sequential hydration
  energies of Na\sur{+}/water and F\sur{-}/water clusters. This builds off an existing approach called the cluster pair approximation but makes no 
  extrathermodynamic assumption to extrapolate from small cluster data to the bulk. I evaluated the energy differences to \emph{n}
  = 200, well past the bulk limit of \emph{n} = 105, using a polarizable force field model. For clusters up to \emph{n} = 35, I was able to compare the
  results with MP2 level quantum chemistry. These results showed a shift in the sequential hydration enthalpy not observed by the cluster pair approximation,
  which assumed the small differences in these values for small clusters remained small in the limit of \emph{n} $\rightarrow \infty$. The shift reflected
  preferential anion solvation with the addition of the second hydration shell. The shift is supported by even higher level calculations at the CCSD(T)
  level of theory. The deviation from the reportedly \emph{ideal} behavior of this ion pair settled around -8 kcal/mol. A simple rearrangement in the
  expression for the solvation enthalpy allowed me to solve for the electrochemical surface potential of water at -0.4 V. Several indirect evidences
  support this figure over the recommended literature value of +0.13 V which is an average of values ranging from -0.5 V to +3--4 V in the literature.
  I argued this average is a compilation of two different surface potential contributions ($\phi\sous{np}$ and $\phi\sous{sp}$) and so is nonphysical.
  My work also supports a small surface potential temperature derivative which is supported by an overlooked paper by Gabdoulline et al.\cite{gabdoulline1997mean}.
  
  Recalling that $\phi\sous{np}$ = $\phi\sous{sp}$ + $\phi\sous{lp}$, in the last chapter of this thesis I show that a local interfacial potential
  $\phi\sous{lp}$ contributes to ion solvation properties even in periodic boundary conditions. This potential depends on the solvent and force field 
  used to simulate it. I illustrated its influence by calculating the solvation free energy of large spherical ions approaching the size of the
  tetraphenyl arsonium and tetraphenyl borate ions. The large size and screening ligands are believed to mask the sign of the charge at the core, giving 
  these ions equal solvation free energies in \emph{all} solvents. Previous studies had shown water to significantly favor the anion by approximately
  20 kcal/mol. I used quasichemical theory to partition the solvation free energy into an ion nonspecific packing contribution, ion specific inner shell
  or chemical contribution, and a long ranged term. The long ranged term contains contributions from the vdW potential and if conditioned properly an
  accurately Gaussian electrostatic part. I showed that the mean-field term in a cumulant expansion of this energy, which is associated with the local
  potential, was the source of the disparity between the anion and cation solvation free energies. By removing it, I found that the tetraphenyl arsonium 
  and tetraphenyl borate ions shared nearly identical solvation properties in SPC/E modeled water and OPLS/AA modeled dimethyl sulfoxide. However, 
  an anion favoring asymmetry persisted in the low dielectric 1,2-dichloroethane solvent also modeled with OPLS/AA. This result has profound implications
  in the force field development community as it brings to light the source of a glaring problem which has led to a great amount of inconsistency in 
  the fitting of parameters. My work almost demands that we as a community attempt some level of standardization as to how we work with these potentials,
  how changing parameters to match this experiment or that (which can differ by the presence of these potentials) affects the solvation structure, and
  how best to relate simulation to experiment so that the symbiotic relationship developed between the approaches can be maintained.
  
  \vspace{12mm}  
  
  \begin{wbepi}{\#LOTRYourResearch}
   It's a dangerous business, Sodium, crossing the air/water interface. You step into the solvent, and if you don't know the real surface potential, 
   there's no knowing where you might be swept off to.
  \end{wbepi}
 
  \vspace{6mm}
  
  High-five, you did it!
  
  \vspace{6mm}

  \begin{wbepi}{Philip J. Fry, Futurama}
   Indeed so. Most indeedly.
  \end{wbepi}  
  
  \vspace{6mm}
  
  In triumph, we sing!
  
  \vspace{6mm}  
  
  \begin{wbepi}{Tom Servo, Mystery Science Theater 3000}
   Oh baby, Rowsdower saves us and saves all the world!
  \end{wbepi}  
  
\end{conclude}