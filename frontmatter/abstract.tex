\textbf{}

\svnidlong{$LastChangedBy$}{$LastChangedRevision$}{$LastChangedDate$}{$HeadURL: http://freevariable.com/dissertation/branches/diss-template/frontmatter/abstract.tex $}
\vcinfo{}

The establishment of a single-ion thermodynamic scale is essential to addressing the ubiquitous specific ion effects observed in the literature. Solvation free energies,
enthalpies, and entropies when decomposed to their single-ion components include a contribution from 1) a solvophobic effect to make room for the ion and the establishment
of ion/solvent interactions and 2) a distant interfacial potential (e.g., the air/water interface). Since experiment can only access the pair quantities, assumptions made
to break the figures into a single-ion scale tend to clump into one of two scales. This thesis makes the case that the two scales reflect 1) (``\emph{bulk}'') and 1) + 2)
(``\emph{real}''). Both 1) and 2) pose a significant challenge to theoretical characterization. The source of the difficulty for the \emph{bulk} thermodynamic scale is in
the handling of non-electrostatic forces between the ion and solvent. My results indicate that polarization plays an important role in ion/water clusters and in the modeling
of energy storage solvents like ethylene and propylene carbonate which are very polarizable molecules. My work also draws attention to the strength of dispersion interactions
between anions and water, as well as, the almost fluid-like nature of the excess electron's density in anions. About 20\% of the charge is diffused over waters in the first
solvation shell. Fortunately these effects are found to be relatively short-ranged (1--2 solvation shells) inviting the possibility of simpler models handling longer-ranged
interactions. The \emph{real} scale on the other hand adds the surface potential experienced by a charge crossing the air/water interface. This is found to be -q0.4 V, 
where q is the signed ion charge, resolving a century old problem. It is further argued that the surface potential contains two contributions a) across the air/water junction 
and b) across the local ion/solvent boundary. The latter of these is present even in simulations using periodic boundary conditions (PBC). My work re-interprets the findings
of previous PBC simulations of large oppositely charged ions which should have equal solvation free energies on scale 1) in water, dimethyl sulfoxide, and 1,2-dichloroethane.
It is found that the simulated result without any conditioning predicts large asymmetries in the free energies of the ions favoring the anion in water and 1,2-dichloroethane 
and the cation in dimethyl sulfoxide. Quasichemical theory is used to remove the contribution made by b) so that the result falls on thermodynamic scale 1). The asymmetries 
noted before are substantially reduced in both water and dimethyl sulfoxide consistent with the findings of others. Parity in these figures is not observed in 1,2-dichloroethane.
Implications for the force field development community are discussed.

\vspace{6mm}


